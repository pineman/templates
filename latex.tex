\documentclass[portuguese, a4paper, titlepage]{article}
\usepackage[margin=2.5cm]{geometry}
%\usepackage{fontspec} % XeLaTeX
\usepackage[T1]{fontenc} % LaTeX
\usepackage[utf8]{inputenc}
%\usepackage{newtxmath, newtxtext}
%\usepackage{lmodern}
\usepackage{csquotes}
\usepackage{babel}
\usepackage[backend=bibtex]{biblatex}
%\usepackage[backend=biber]{biblatex}
\addbibresource{bibliography.bib}

\usepackage{indentfirst}
\usepackage{graphicx}
	\graphicspath{{images/}}
\usepackage{grffile}
\usepackage{float}
\usepackage{amsmath}
	\allowdisplaybreaks
\usepackage{commath}
\usepackage{amssymb}
\usepackage{mathtools}
\usepackage{siunitx}
	\sisetup{inter-unit-product =\ensuremath{.}}
\usepackage{hyperref}

% Section style
\renewcommand{\thesection}{\Roman{section}}
\renewcommand{\thesubsection}{\alph{subsection})}
\renewcommand{\thesubsubsection}{\roman{subsubsection}.}

% Useful commands
\newcommand{\eq}{\Leftrightarrow} % Equivalente
% Ordem de grandeza, e.g., "2\og{5}" => "2e5"
\newcommand{\og}[1]{{\times \num{e#1}}}
% Em um ponto, e.g. "f(x)\at{x=5}" = f(x)|x=5
\newcommand{\at}[1]{\left.\right|_{#1}}
 % Para numerar apenas uma equação
\newcommand\numberthis{\addtocounter{equation}{1}\tag{\theequation}}
\DeclareMathOperator{\Div}{div}
\DeclareMathOperator{\Rot}{rot}
\newcommand{\real}{\ensuremath{\mathds{R}}}
 % Operator "d", e.g., "\frac{\dx{f}}{\dx{x}} = "df/dx"
\newcommand{\dx}[1]{\ensuremath{\operatorname{d}\!{#1}}}

% Header and footer
\usepackage{fancyhdr}
\pagestyle{fancy}
\fancyhf{}
\lhead{left header}
\rhead{right header}
\lfoot{left footer}
\rfoot{Página \thepage}
\renewcommand{\headrulewidth}{1.0pt}
\renewcommand{\footrulewidth}{0.5pt}

% Document
\begin{document}
	\hypersetup{pageanchor=false}
	\begin{titlepage}

\center

\textsc{\bfseries\LARGE Instituto Superior Técnico}\\[1cm] % Name of your university/college
\includegraphics[height=1.5cm]{IST_Logo.pdf}\\[5cm]

\textsc{\large Engenharia Eletrotécnica e de Computadores}\\[0.5cm] % Major heading such as course name
\textsc{\Large Eletrotecnia Teórica}\\[0.5cm] % Minor heading such as course title
\textsc{\large 2016/2017 - 2º Ano - 2º Semestre}\\

\rule{\textwidth}{1.6pt}\vspace*{-\baselineskip}\vspace*{2pt} % Thick horizontal line
\rule{\textwidth}{0.4pt}\\[\baselineskip] % Thin horizontal line

{\Huge \bfseries 4º Trabalho Laboratorial}\\[0.2cm]
\bigskip
{\huge \bfseries Regimes transitórios}\\[0.2cm]

\rule{\textwidth}{0.4pt}\vspace*{-\baselineskip}\vspace{3.2pt} % Thin horizontal line
\rule{\textwidth}{1.6pt}\\[5cm]

\begin{minipage}{0.9\textwidth}
	\begin{flushleft} \large
		\begin{Large}\bfseries\textsc{Autores:}\end{Large}\\[0.4cm]
		\begin{tabular}{l l l}
			André Agostinho & 84001 & \normalsize andre.f.agostinho@tecnico.ulisboa.pt \\
			João Morais			& 84080 & \normalsize	joao.de.morais@tecnico.ulisboa.pt \\
			João Pinheiro		& 84086 & \normalsize joao.castro.pinheiro@tecnico.ulisboa.pt \\
			João Freitas		& 84093 & \normalsize joao.m.freitas@tecnico.ulisboa.pt \\
			Thomas Berry		& 84189 & \normalsize thomasdpberry@tecnico.ulisboa.pt \\
		\end{tabular}
	\end{flushleft}
\end{minipage}\\[0.5cm]

\large \bfseries Laboratório sexta-feira, 14h00-16h00\\
\large 19 de maio de 2017\\[1cm]

\end{titlepage}

	\hypersetup{pageanchor=true}

	hello

	%\begin{figure}[h]
	%	\centering
	%	\includegraphics[width=0.6\linewidth]{figure.pdf}
	%	\caption{An example figure}
	%	\label{fig:figure-example}
	%\end{figure}

	Ver \autoref{fig:figure-example}
\end{document}
